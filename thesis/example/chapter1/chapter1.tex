
\chapter{~~~ 序論}
\label{ch:ch1}

\section{~全人教育的由來 }
\label{se:se1}
在概念上,「全人教育」原本就是我國傳統教育的目標,然而面對科技發展的影響與衝擊,現代人所經歷的教育環境往往只能塑造一個在專業領域學有專精的單面向人(One dimension person),而無法教育出具多方面智慧和能力、有完整人格的人。\par
換言之,這種「偏頗的教育」所造就出來的人,固然可以滿足社會發展的需求,但無法善盡一個「人」的職責;因此,需要有一種整全而平衡的教育,塑造完整人格,以協助現代人重建圓融美滿的人生,這就是「全人教育」的由來。

\section{~理念篇}
\label{se:se2}
中原大學的全人教育和「教育宗旨與理念」有什麼關聯 ? 
\par
在思考制訂本校的教育宗旨及教育理念的過程中,發現我們的整體思維, 無論在教育宗旨、、倫理觀念、教育目標、教育功能、教育動力、學術立場、 教育方式或傳統價值,無一不是指向培培養「有信、有望、有愛」,兼備專業知識、 品格涵養和生命智慧的知識份子之目標。因此,我們推動的「全人教育」,和本校的教育宗旨與理念息息相關,緊密連結。
\par
我們的教育理念共七條,可分述如下:
\\
\par
「我們尊重自然與人性的尊嚴,尋求天人物我間的和諧,以智慧慎用科技與人文的專業知識,造福人群。」〈倫理觀念〉 
\\\\
「我們瞭解人人各承不同之秉賦,其性格、能力與環境各異,故充分發揮個人潛力就是成功。」 〈教育目標〉 
\\\\
「我們認為教育不僅是探索知識與技能的途徑,也是塑造人格、追尋自我生命意義的過程。」 〈教育功能〉 
\\\\
「我們確信『愛』是教育的主導力量,願以身教言教的方式,互愛互敬的態度,師生共同追求成長。」 〈教育動力〉
\\\\
「我們尊重學術自由與自主,並相信知識使人明理,明理使人自由。」 〈學術立場〉
\\\\
「我們相信踐履篤實的教育方式是尋求真知的途徑。」 〈教育方式〉
\\\\
「我們深以虔敬上主、摯愛國家、敬業樂群、崇尚簡樸的傳統校風為榮。」 〈傳統價值〉
\\\\

\section{~內涵篇}
\label{se:se3}
中原大學全人教育的核心內涵是什麼? 
\par
中原大學教育理念之首要,特別強調「我們尊重自然與人性的尊嚴,尋求天人物我間的和諧,以智慧慎用科技與人文的專業知識,造福人群。」因此,「天」、「人」、「物」、「我」即為本校全人教育理念之核心內涵。
\par
全人教育中的「天」指的是什麼? 
\par
所謂「天」,乃指活在關係中的人,以信仰崇拜、敬天愛神的態度追求終極意義,探索永恆價值。自古以來人類都一直為生死課題、人生價值、天堂地獄、宇宙探源等所謂終極關懷問題而苦尋答案。此為人生意義之追求所必須面對的重要課題。 
\par
全人教育中的「人」指的是什麼? 
\par
所謂「人」是指活在屬於人的各種關係價值之中,而不是一個完全離群索居的「人」。依社會學的意義而言,甚難將這種離群索居的人稱之為「人」。當今所謂的現代人,常極力否定人與其他事物之關係,在他們的心目中,認為任何關係都是一種束縛,都是對個人自由的斲傷。這種思想產生了否定一切的心態,拒絕權威,拒絕真理,完全以個體自由為標準。然而事實上,人是有限的,人是不足的,作為一個勇敢的人,必須面對自己的一切,知道自己有限、不足,有「是什麼就是什麼」的勇氣,才有更求上進的意志,田立克 (Paul Tillich) 所謂「courage to be」,正是此意。 
\par
全人教育中的「物」指的是什麼 ? 
\par
天生萬物必有我用,人與物的關係應是一種「共生」的關係,人利用各種技術和物質以改善生活的環境,促進人類的幸福。所謂「人」為萬「物」之靈,此之謂也。但是,不幸的是今天的人已為「物」所役,甚至與「物」相對峙,人的主體性完全喪失,今天的世界似乎已是追求「物」為萬「人」之靈的世界,此為何等悲哀之事。
\par 
 全人教育中的「我」指的是什麼?
 \par 
「我」是獨特的 (unique),生命是不可取代的。然而人生在世也絕對不可能完全孑然獨存,不受任何人之影響,做一個「人」必須有勇氣面對自己的有限、軟弱與無能,如此,方能繼續努力、發揮潛能,盡一己之力以完成個體存在之意義。
