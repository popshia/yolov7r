
\chapter{~~~ 論文格式}
\label{ch:ch1}

\section{~自動為圖片、表格增加序號 }
\label{se:se1}
\subsection{標號}
插入圖片之後,在要加上編號的地方,點選「參考資料」/「標號」/「插入標號」。第一次新增標號時,要點選「新增標籤」,輸入「圖」或「表」,之後只要在標籤裡選擇「圖」或「表」即可,如果有設定好大綱編號,可以在編號方式中將「包含章節編號」打勾,在這個範本第幾章是使用標題1,所以章節起始樣式選標題1,按確定,插入標號後,按一下tab (不要使用空白,以免影響到圖目錄的對齊),接著在後面打上想要的說明即可。
\begin{figure}
 \centering
 \includegraphics[width=.8\textwidth]{figure/figure1.png} %1.png是图片文件的相对路径
 \caption{插入標號的路徑} %標題
 \label{img} 
\end{figure}
\begin{figure}
 \centering %置中
 \includegraphics[width=.8\textwidth]{figure/figure2.png}
\caption{標號設定}
\end{figure}
\clearpage
\begin{table}[]
\caption{測試表格}
\begin{tabular}{|c|c|c|}
\hline
\rowcolor[HTML]{D9D9D9} 
\begin{tabular}[c]{@{}c@{}}各院名稱\\ \\ College of Name\end{tabular}                                           & \begin{tabular}[c]{@{}c@{}}精裝論文顏色\\ \\ Color(tooled in gold font\end{tabular}                           & 顏色參考 \\ \hline
\begin{tabular}[c]{@{}c@{}}理學院\\ \\ College of Science\end{tabular}                                         & {\color[HTML]{000000} \begin{tabular}[c]{@{}c@{}}{\color{blue}藍色} (燙金字體)\\ \\ {\color{cyan}Blue}(tooled in gold font)\end{tabular}} & 1\#  \\ \hline
\begin{tabular}[c]{@{}c@{}}工學院\\ \\ College of Engineering\end{tabular}                                     & \begin{tabular}[c]{@{}c@{}}黑色(燙金字體)\\ \\ Black(tooled in gold font)\end{tabular}                        &      \\ \hline
\begin{tabular}[c]{@{}c@{}}商學院\\ \\ College of Business\end{tabular}                                        & \begin{tabular}[c]{@{}c@{}}{\color{green}綠色}(燙金字體)\\ \\ {\color{green}Green} (tooled in gold font)\end{tabular}                        & \#   \\ \hline
\begin{tabular}[c]{@{}c@{}}設計學院\\ \\ College of Design\end{tabular}                                         & \begin{tabular}[c]{@{}c@{}}{\color{brown}咖啡色}(燙金字體)\\ \\ {\color{brown}Brown}(tooled in gold font)\end{tabular}                       & \#   \\ \hline
\begin{tabular}[c]{@{}c@{}}人育學院\\ \\ College of Humanities and Education\end{tabular}                       & \begin{tabular}[c]{@{}c@{}}{\color{red}紅色}(燙金字體)\\ \\ {\color{red}Red}(tooled in gold font)\end{tabular}                          & \#   \\ \hline
\begin{tabular}[c]{@{}c@{}}電資學院\\ \\ College of Electrical Engineering and \\ Computer Science\end{tabular} & \begin{tabular}[c]{@{}c@{}}{\color{darkgray}淺銀灰藍色}(燙金字體)\\ \\ {\color{darkgray}Silver gray blue}(tooled in gold font)\end{tabular}          & \#   \\ \hline
\begin{tabular}[c]{@{}c@{}}法學院\\ \\ School of Law\end{tabular}                                              & \begin{tabular}[c]{@{}c@{}}{\color{red}暗紅色}(燙金字體)\\ \\ {\color{red}dark red}(tooled in gold font)\end{tabular}                    & \#   \\ \hline
\end{tabular}

\end{table}
\clearpage
\subsection{標題及內文格式}
中文字型若為標楷體、西文字型則為 Times New Roman
\begin{table}[]
\caption{格式說明}
\begin{tabular}{|l|l|l|}
\hline
類別  & 字體大小 & 編號方式 \\ \hline
內文  & 12   & 無    \\ \hline
標題一 & 18   & 第*章  \\ \hline
標題二 & 16   & 第*節  \\ \hline
標題三 & 14   & *、   \\ \hline
圖表  & 9    & 標號   \\ \hline
\end{tabular}
\end{table}
\clearpage
\section{~新增圖、表目錄 }
\label{se:se2}
 在「參考資料」之「標號」中,點選插入「圖表目錄」,然後選擇「標題標籤」即可。

\begin{figure}[b]
 \centering %置中
 \includegraphics[width=.8\textwidth]{figure/figure3.png}
\caption{插入圖表目錄}
\end{figure}