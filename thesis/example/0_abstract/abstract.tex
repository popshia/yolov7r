\chapter{\begin{center}摘要\end{center}}
\label{ch:abstract}
民國42年由篤信基督、熱心教育人士張靜愚先生、郭克悌先生、賈嘉美牧師、鈕永健先生、陳維屏先生、瞿荊洲先生及桃園中壢地方士紳吳鴻森先生、徐崇德先生等會商籌設一所農工學院,期以基督救世愛人的精神,為國家造就高深科學與工程人才;在歷經多次籌備更名,於民國44年10月奉教育部核准立案,定名為「私立中原理工學院」,以「篤信力行」為校訓,設物理、化學、化學工程、土木工程等4個學系;民國69年8月1日改制為中原大學。
\par
中原大學已走過半個多世紀的歲月,在歷屆董事會支持下,經郭克悌、謝明山、韓偉、阮大年、尹士豪、張光正、熊慎幹、程萬里、張光正等諸位校長掌理校務,貢獻良多,厥功甚偉,並分別代表了中原初創、奠基、成長、茁壯、擴張、拓展及新象等階段。現今擁有理、工、商、法、設計、人文與教育、電機資訊等7個學院、29個學士班(含27個學系、1個電資學院學士班、1個原住民專班)、38個碩士班、13個博士班及19個碩專班;歷屆畢業校友已達13萬餘人,在國家重大建設中,竭盡心力貢獻所學,深獲各界讚譽。 
\par
瞻望未來,中原除在既有的教育宗旨與理念之精神憲法下,建構「三創教育」-活用創意、激發創新、迎向創業之特色競爭優勢為標的;揭櫫「全人教育」、「生命關懷」及「服務學習」為發展目標;積極培育學生品格精進及國際視野;並以穩健踏實的辦學理念為國內高等教育盡一份心力,全校師生亦在各項表現中展現傑出優異的成果,屢獲產官學界給予肯定之耀眼成績。本校持續朝向以具有世界知名度並擁有諸多國內標竿特色之「有信、有望、有愛」的卓越大學邁進。
